\documentclass{article}
\usepackage[utf8]{inputenc}
\usepackage[T1]{fontenc}
\usepackage{url}
\usepackage{graphicx}
\usepackage{geometry}
\usepackage{listings}
\usepackage{hyperref}
\usepackage{multicol}
\usepackage{array}
\usepackage{float}
\usepackage{booktabs}
\usepackage{xcolor}
\usepackage{tcolorbox}
\lstset{numbers=none, basicstyle=\ttfamily}
\renewcommand{\lstlistingname}

\geometry{a4paper, left=20mm, right=20mm, top=20mm, bottom=20mm}
\title{Clickjacking (UI redressing) Vulnerability Notes}
\author{Eyad Islam El-Taher}

\begin{document}

\maketitle

\section*{Introduction}
Clickjacking, also known as a "UI Redress Attack," is an interface-based malicious attack. The core principle is deceptively simple: an attacker tricks a user into clicking on a hidden or disguised element on a webpage. The user believes they are interacting with a visible, harmless page (e.g., playing a game or watching a video), but their click is secretly hijacked to perform an unintended action on a different, hidden website or application.\\

It is essentially a violation of the user's intent, exploiting their trust and logged-in sessions to execute unauthorized commands.

\begin{figure}[H]
    \centering
    \includegraphics[width=0.8\textwidth]{clickjacking-diagram.png}
    \label{fig:clickjacking}
\end{figure}

\section*{Clickjacking Vulnerability Happen When}
\begin{enumerate}
    \item \textbf{The target website lacks framing protections}:
    \begin{itemize}
        \item Missing \texttt{X-Frame-Options} header
        \item No \texttt{Content-Security-Policy} with \texttt{frame-ancestors} directive
        \item Weak or bypassable frame-busting scripts
    \end{itemize}
    
    \item \textbf{The application has click-triggered sensitive actions}
    
    \item \textbf{The user has an active authenticated session}
    
    \item \textbf{Social engineering convinces the user to click}
\end{enumerate}

\section*{Clickjacking Vulnerability Happen Where}
\begin{itemize}
    \item \textbf{Social Media Platforms}: Unauthorized liking/sharing/following
    \item \textbf{Webmail Clients}: Forced email deletion/composition
    \item \textbf{Online Banking}: Unauthorized fund transfers
    \item \textbf{Administrative Panels}: Configuration changes
    \item \textbf{E-commerce Sites}: Unwanted purchases/cart modifications
    \item \textbf{Account Management Pages}: Password/email changes
    \item \textbf{Browser Permission Dialogs}: Camera/microphone access
\end{itemize}

\section*{Impact of Clickjacking Vulnerability}
\begin{itemize}
    \item \textbf{Financial Loss}: Unauthorized bank transfers and transactions
    \item \textbf{Account Takeover}: Password and email changes
    \item \textbf{Data Breach}: Unauthorized data access and sharing
    \item \textbf{Reputation Damage}: Social media actions performed without consent
    \item \textbf{Privacy Violation}: Unauthorized access to webcam/microphone
    \item \textbf{Business Disruption}: Administrative settings changes
\end{itemize}



\hrule

\section*{X-Frame \& Content-Security-Policy and Their Roles}

\subsection*{X-Frame-Options Header}
The \texttt{X-Frame-Options} HTTP header is a legacy but widely supported defense mechanism that instructs the browser whether a page can be embedded in a frame, iframe, embed, or object.

\begin{tcolorbox}[title=X-Frame-Options Directives]
\begin{itemize}
    \item \textbf{DENY} - Prevents any framing whatsoever
    \item \textbf{SAMEORIGIN} - Allows framing only by pages from the same origin
    \item \textbf{ALLOW-FROM https://example.com} - Allows framing only by specified domain (poor browser support)
\end{itemize}
\end{tcolorbox}


\subsection*{Content-Security-Policy (CSP) Header}
The \texttt{Content-Security-Policy} header is the modern replacement for \texttt{X-Frame-Options}, offering more granular control through the \texttt{frame-ancestors} directive.

\begin{tcolorbox}[title=CSP frame-ancestors Directives]
\begin{itemize}
    \item \textbf{frame-ancestors 'none';} - Equivalent to \texttt{DENY}
    \item \textbf{frame-ancestors 'self';} - Equivalent to \texttt{SAMEORIGIN}
    \item \textbf{frame-ancestors https://trusted.com;} - Allows specific domains
    \item \textbf{frame-ancestors 'self' https://*.example.com;} - Complex domain patterns
\end{itemize}
\end{tcolorbox}


\section*{X-Frame \& Content-Security-Policy Misconfigurations}

\subsection*{Common X-Frame-Options Misconfigurations}

\begin{tcolorbox}[title=Critical Misconfigurations]
\begin{enumerate}
    \item \textbf{Header Not Set} - Complete absence of X-Frame-Options header
    \item \textbf{Inconsistent Application} - Header set on some pages but not others
    \item \textbf{Using ALLOW-FROM} - Poor browser support makes this ineffective
    \item \textbf{Case Sensitivity} - Incorrect casing (e.g., \texttt{sameorigin} vs \texttt{SAMEORIGIN})
    \item \textbf{Multiple Headers} - Conflicting X-Frame-Options directives
\end{enumerate}
\end{tcolorbox}

\subsection*{Common CSP frame-ancestors Misconfigurations}

\begin{tcolorbox}[title=Dangerous CSP Patterns]
\begin{enumerate}
    \item \textbf{Wildcard Usage} - \texttt{frame-ancestors *;} (Allows all domains)
    \item \textbf{Overly Permissive} - \texttt{frame-ancestors 'self' https://*;} 
    \item \textbf{Missing Directive} - CSP present but without \texttt{frame-ancestors}
    \item \textbf{Report-Only Mode} - Using \texttt{Content-Security-Policy-Report-Only} without enforcement
    \item \textbf{Inheritance Issues} - Child frames without proper CSP inheritance
\end{enumerate}
\end{tcolorbox}

\textbf{Vulnerable Examples:}
\begin{lstlisting}[language=bash]
# VULNERABLE: Allows framing from any HTTPS domain
Content-Security-Policy: frame-ancestors https:;

# VULNERABLE: Wildcard allows all domains
Content-Security-Policy: frame-ancestors *;

# VULNERABLE: Missing frame-ancestors directive
Content-Security-Policy: default-src 'self';
\end{lstlisting}

\hrule

\section*{How to Identify Clickjacking Vulnerabilities}

\subsection*{1- Manual Testing Methods}

\begin{tcolorbox}[title=Basic Manual Test Template]
I created an HTML file with the following code to test if a target page can be framed. The file name is \texttt{clickjacking-test.html}  just open it in a browser.
\end{tcolorbox}

\subsection*{Usage Instructions:}

\begin{enumerate}
    \item Save the code above as \texttt{clickjacking-test.html}
    \item Edit the \texttt{src} attribute in the iframe to point to your target URL
    \item Adjust the CSS positions to align the overlay with sensitive buttons
    \item Open the file in a web browser while logged into the target application
    \item Click the red overlay button to test if actions are triggered in the background frame
\end{enumerate}

\subsection*{2- Automated Tools \& Browser Extensions}

\begin{tcolorbox}[title=Firefox Addon: Click-jacking by Daoud Youssef]
This extension provides immediate visual feedback about clickjacking vulnerabilities:
\begin{itemize}
    \item \textbf{Functionality}: Adds a red border to webpages vulnerable to clickjacking
    \item \textbf{Detection Method}: Identifies missing \texttt{X-Frame-Options} header
    \item \textbf{Usage}: Simply browse to any webpage - visual indicators show vulnerability status
    \item \textbf{Benefit}: Quick, real-time assessment during penetration testing
\end{itemize}
\end{tcolorbox}

\textbf{How the Extension Works:}
\begin{enumerate}
    \item Install the extension from Firefox Add-ons store
    \item Browse to target website
    \item If page loads with \textbf{red border} = Vulnerable (missing X-Frame-Options)
    \item If no red border = Protected (X-Frame-Options header present)
\end{enumerate}

\subsection*{3- Command-Line Identification Methods}

\begin{lstlisting}[language=bash]
# Basic header check
curl -I https://example.com/sensitive-page

# Check specifically for security headers
curl -I https://example.com | grep -i "x-frame-options\|content-security-policy"

# Comprehensive security header check
curl -s -I https://example.com | grep -E "(X-Frame-Options|Content-Security-Policy|X-Content-Type-Options)"
\end{lstlisting}

\section*{False Positive Checks}

\subsection*{Common False Positive Scenarios}

\begin{tcolorbox}[title=False Positive Scenarios to Consider]
\begin{enumerate}
    \item \textbf{JavaScript Frame Busting}: Page lacks headers but has client-side frame protection
    \item \textbf{Dynamic Content}: Pages that shouldn't be framed (error pages, static content)
    \item \textbf{Intentional Framing}: Public pages designed to be embedded (widgets, help pages)
    \item \textbf{Report-Only Mode}: CSP in report-only without enforcement
\end{enumerate}
\end{tcolorbox}

\subsection*{Comprehensive Assessment Checklist}

\begin{tcolorbox}[title=Clickjacking Assessment Checklist]
\begin{itemize}
    \item  Check for \texttt{X-Frame-Options} header presence and value
    \item  Check for \texttt{Content-Security-Policy} with \texttt{frame-ancestors}
    \item  Verify header consistency across all sensitive pages
    \item  Test actual framing capability with proof-of-concept
    \item  Verify if framing is intentional for public content
    \item  Test across different user roles (public, authenticated, admin)
    \item  Check for inheritance in subdomains and child frames
\end{itemize}
\end{tcolorbox}

\subsection*{Burp Suite Integration}

\begin{lstlisting}
 Using Burp Scanner checks for clickjacking:
 1. Enable "Check for missing X-Frame-Options header" in scan configuration
 2. Use "Embedded content" checks in engagement tools

\end{lstlisting}

\section*{Burp Suite Clickbandit}

\subsection*{What is Burp Clickbandit?}

\begin{tcolorbox}[title=Burp Clickbandit Overview]
Burp Clickbandit is a tool that helps test for clickjacking vulnerabilities by creating exploit proof-of-concepts directly from your browser.
\end{tcolorbox}

\textbf{Key Features:}
\begin{itemize}
    \item Creates clickjacking attack HTML files automatically
    \item Works directly in your browser using JavaScript
    \item Records user actions to build realistic exploits
    \item Bypasses frame busters with sandboxing
    \item Generates ready-to-use attack pages
\end{itemize}

\subsection*{Setup Steps}

\begin{enumerate}
    \item \textbf{Open Clickbandit} in Burp Suite:
    \begin{itemize}
        \item Go to \texttt{Burp Menu → Burp Clickbandit}
    \end{itemize}
    
    \item \textbf{Copy the script}:
    \begin{itemize}
        \item Click \texttt{Copy Clickbandit to clipboard}
    \end{itemize}
    
    \item \textbf{Visit target website} in your browser
    
    \item \textbf{Open Developer Console}:
    \begin{itemize}
        \item Press \texttt{F12} or right-click → \texttt{Inspect Element}
        \item Go to \texttt{Console} tab
    \end{itemize}
    
    \item \textbf{Paste and run} the Clickbandit script
\end{enumerate}

\subsection*{Running an Attack}

\begin{tcolorbox}[title=Attack Workflow]
Once Clickbandit banner appears at the top of your browser:
\end{tcolorbox}

\begin{enumerate}
    \item Click \texttt{Start} to begin recording
    
    \item \textbf{Perform actions} on the website:
    \begin{itemize}
        \item Click buttons, links, forms
        \item Mimic what a victim would do
        \item All actions are recorded
    \end{itemize}
    
    \item Click \texttt{Finish} when done
    
    \item \textbf{Optional settings}:
    \begin{itemize}
        \item \texttt{Disable click actions} - Prevents real clicks on target site
        \item \texttt{Sandbox iframe} - Bypasses frame busters
    \end{itemize}
\end{enumerate}

\subsection*{Reviewing and Saving}

\begin{tcolorbox}[title=Review Tools]
After completing the attack, use these features to test and save:
\end{tcolorbox}

\textbf{Review Commands:}
\begin{itemize}
    \item \texttt{Toggle transparency} - Show/hide original page
    \item \texttt{Reset} - Restore attack to initial state
    \item \texttt{+/- buttons} - Zoom in/out
    \item \texttt{Arrow keys} - Reposition attack UI
\end{itemize}

\textbf{Save the exploit:}
\begin{itemize}
    \item Click \texttt{Save} to download HTML file
    \item This file is your clickjacking proof-of-concept
    \item Use it to demonstrate the vulnerability
\end{itemize}

\subsection*{Clickbandit Workflow Summary}

\begin{lstlisting}
1. Burp Menu -> Burp Clickbandit
2. Copy script to clipboard
3. Browse to target site
4. Open Developer Console (F12)
5. Paste script and press Enter
6. Click "Start" in Clickbandit banner
7. Perform actions on target site
8. Click "Finish"
9. Review and test the attack
10. Click "Save" to download HTML PoC
\end{lstlisting}



\hrule

\section*{Exploitation Techniques \& Notes}

\subsection*{1- \underline{Basic Clickjacking Exploit}}
\begin{lstlisting}[basicstyle=\ttfamily\small, frame=single]
<head>
	<style>
		#target_website {
			position:relative;
			width:128px;
			height:128px;
			opacity:0.00001;
			z-index:2;
			}
		#decoy_website {
			position:absolute;
			width:300px;
			height:400px;
			z-index:1;
			}
	</style>
</head>
...
<body>
	<div id="decoy_website">
	...decoy web content here...
	</div>
	<iframe id="target_website" src="https://vulnerable-website.com">
	</iframe>
</body>
\end{lstlisting}

\subsection*{2- \underline{Clickjacking with Prefilled Form Input}}

\subsection*{The Vulnerability}

\begin{tcolorbox}[title=Prefilled Form Clickjacking]
Websites that allow form prepopulation via GET parameters are vulnerable to advanced clickjacking attacks where form values are preset before the user clicks.
\end{tcolorbox}

\textbf{How it works:}
\begin{itemize}
    \item Target website accepts form prepopulation via URL parameters
    \item Example: \texttt{https://site.com/form?email=attacker@evil.com}
    \item Attacker creates clickjacking page with prefilled target URL
    \item Victim only needs to click "Submit" - all form data is already set
\end{itemize}

\subsection*{Attack Scenario}

\begin{lstlisting}
https://vulnerable-bank.com/transfer?to_account=ATTACKER_123&amount=1000
\end{lstlisting}
Victim only needs to click "Confirm Transfer"

\subsection*{3- \underline{Exploitation: Frame Busting Scripts}}

\subsection*{What are Frame Busting Scripts?}

\begin{tcolorbox}[title=Frame Busting Definition]
Client-side protection scripts that prevent a website from being loaded inside frames or iframes, typically implemented through browser extensions or JavaScript.
\end{tcolorbox}

\textbf{Common Frame Busting Behaviors:}
\begin{itemize}
    \item Check if current window is the main/top window
    \item Make all frames visible to the user
    \item Prevent clicking on invisible frames
    \item Intercept and alert users about potential clickjacking attacks
\end{itemize}

\subsection*{Bypassing Frame Busting Scripts}

\begin{tcolorbox}[title=Attackers' Workarounds]
Frame busting scripts have limitations that attackers can exploit.
\end{tcolorbox}

\textbf{Vulnerabilities in Frame Busting:}
\begin{itemize}
    \item \textbf{Browser-specific}: Techniques vary across browsers
    \item \textbf{JavaScript dependency}: Requires JavaScript enabled
    \item \textbf{HTML flexibility}: Multiple ways to circumvent protection
    \item \textbf{Security settings}: Browser settings may block the scripts
\end{itemize}

\subsection*{HTML5 Sandbox Bypass}

\begin{tcolorbox}[title=Effective Bypass Technique]
Using HTML5 iframe \texttt{sandbox} attribute to neutralize frame busters.
\end{tcolorbox}

\begin{lstlisting}[basicstyle=\ttfamily\small, frame=single]
<iframe id="victim_website" 
        src="https://victim-website.com" 
        sandbox="allow-forms allow-scripts">
</iframe>
\end{lstlisting}

\textbf{How this bypass works:}
\begin{itemize}
    \item \texttt{allow-forms}: Enables form submission within iframe
    \item \texttt{allow-scripts}: Allows JavaScript execution
    \item \textbf{Missing \texttt{allow-top-navigation}}: Prevents top-level navigation
    \item Frame buster scripts run but cannot navigate the top window
    \item Website remains trapped in the iframe
\end{itemize}

\textbf{\underline{Other Bypass Techniques:}}
\begin{itemize}
    \item \textbf{Double framing}: Nested iframes confuse reference checks
    \item \textbf{onBeforeUnload event}: Intercept navigation attempts
    \item \textbf{XSS filters}: Abuse browser XSS protection
    \item \textbf{Referrer checks}: Spoof or manipulate referrer headers
\end{itemize}

\subsection*{4- \underline{Combining Clickjacking with Other Attacks}}

\subsection*{Attack Overview}

\begin{tcolorbox}[title=Clickjacking + DOM XSS = Critical Impact]
While clickjacking alone can cause harm, its true danger emerges when combined with other vulnerabilities like Stored XSS to execute arbitrary JavaScript.
\end{tcolorbox}

\textbf{Attack Flow:}
\begin{itemize}
    \item Identify a DOM XSS vulnerability in the target website
    \item Craft a malicious URL that triggers the XSS payload
    \item Embed the URL in a transparent iframe
    \item Overlay deceptive UI elements
    \item Victim clicks, triggering both the action AND XSS execution
\end{itemize}

\subsection*{Attack Example}
\begin{lstlisting}[basicstyle=\ttfamily\small, frame=single]
<iframe src="https://vulnerable-website.com/feedback?
name=<img src=1 onerror=alert(document.cookie)>&email=attacker@evil.com
&subject=test&message=test#feedbackResult">
</iframe>
\end{lstlisting}

\subsection*{How This Works}

\begin{tcolorbox}[title=Attack Mechanics]
\begin{itemize}
    \item Victim visits attacker's malicious page
    \item Transparent iframe loads the vulnerable feedback form
    \item DOM XSS payload is automatically injected via URL parameters
    \item Victim clicks deceptive button overlaid on the iframe
    \item Click submits the form AND triggers the XSS payload
    \item Attacker's JavaScript executes in the context of the target site
\end{itemize}
\end{tcolorbox}

\vspace{0.5cm}

\subsection*{5- \underline{Multistep Clickjacking}}

\subsection*{What is Multistep Clickjacking?}

\begin{tcolorbox}[title=Multiple Actions Required]
When an attack requires several clicks to complete, like adding items to a cart and then checking out.
\end{tcolorbox}

\subsection*{Real-World Example: Online Shopping}

\textbf{Attack Scenario:}
\begin{itemize}
    \item Victim visits malicious page
    \item Transparent iframe loads shopping website
    \item Attacker overlays multiple deceptive buttons
    \item Victim clicks thinking they're playing a game
    \item Actually adds items to cart and completes purchase
\end{itemize}

\subsection*{How It Works}

\begin{enumerate}
    \item \textbf{Step 1}: Add item to shopping cart
    \item \textbf{Step 2}: Proceed to checkout
    \item \textbf{Step 3}: Confirm payment
    \item \textbf{Step 4}: Place order
\end{enumerate}

\subsection*{Technical Implementation}

\begin{itemize}
    \item Uses \textbf{multiple iframes} or page divisions
    \item Each iframe handles one step of the process
    \item Requires \textbf{precise positioning} of overlay elements
    \item Must maintain user session across all steps
\end{itemize}

\newpage

\section*{Clickjacking Testing Methodology}

\subsection*{Phase 1: Reconnaissance \& Scope Definition}

\begin{tcolorbox}[title=Step 1: Target Identification]
\begin{enumerate}
    \item Identify all sensitive functionalities:
    \begin{itemize}
        \item Login/authentication pages
        \item Administrative interfaces
        \item Financial transactions
        \item Account management pages
        \item Form submissions with sensitive actions
    \end{itemize}
    \item Map application structure and user roles
    \item Note all endpoints that perform state-changing operations
\end{enumerate}
\end{tcolorbox}

\begin{tcolorbox}[title=Step 2: Header Analysis]
\begin{enumerate}
    \item Use automated scanning:
    \begin{lstlisting}[language=bash]
    # Bulk header checking
    curl -I https://target.com/endpoint | 
    grep -i "x-frame-options\|content-security-policy"
    
    \end{lstlisting}
    \item Check for inconsistencies across pages
    \item Verify both \texttt{X-Frame-Options} and \texttt{CSP frame-ancestors}
    \item Note pages missing protection headers
\end{enumerate}
\end{tcolorbox}

\subsection*{Phase 2: Vulnerability Confirmation}

\begin{tcolorbox}[title=Step 3: Basic Framing Test]
\begin{enumerate}
    \item Create basic clickjacking test file:
    \begin{lstlisting}[language=HTML]
    <iframe src="https://target.com/sensitive-page" 
            style="opacity:0.5; width:500px; height:500px;">
    </iframe>
    \end{lstlisting}
    \item Test if page loads in iframe
    \item Check for visual indicators of frame busting
    \item Verify if page functionality remains accessible
\end{enumerate}
\end{tcolorbox}

\begin{tcolorbox}[title=Step 4: Advanced Testing Methods]
\begin{enumerate}
    \item \textbf{Firefox Extension Check}:
    \begin{itemize}
        \item Install "Click-jacking by Daoud Youssef"
        \item Browse to target pages
        \item Note red borders indicating vulnerabilities
    \end{itemize}
    \item \textbf{Burp Clickbandit Assessment}:
    \begin{itemize}
        \item Follow Clickbandit workflow
        \item Test multi-step actions
        \item Generate PoC for complex scenarios
    \end{itemize}
    \item \textbf{Manual PoC Creation}:
    \begin{itemize}
        \item Create custom HTML files for specific functionalities
        \item Test overlay positioning and user interaction
    \end{itemize}
\end{enumerate}
\end{tcolorbox}

\subsection*{Phase 3: Exploitation Techniques}

\begin{tcolorbox}[title=Step 5: Frame Buster Bypass Testing]
\begin{enumerate}
    \item Test HTML5 sandbox bypass:
    \begin{lstlisting}[language=HTML]
    <iframe src="https://target.com" 
            sandbox="allow-forms allow-scripts">
    </iframe>
    \end{lstlisting}
    \item Attempt double framing techniques
    \item Check for JavaScript dependency in frame busting
    \item Test with different browser security settings
\end{enumerate}
\end{tcolorbox}

\begin{tcolorbox}[title=Step 6: Advanced Attack Vectors]
\begin{enumerate}
    \item \textbf{Prefilled Form Testing}:
    \begin{itemize}
        \item Identify forms with GET parameter prepopulation
        \item Craft URLs with malicious preset values
        \item Test if forms submit with attacker-controlled data
    \end{itemize}
    \item \textbf{Multi-step Action Testing}:
    \begin{itemize}
        \item Map complex workflows (e.g., shopping cart → checkout)
        \item Create multi-iframe attacks
        \item Test session persistence across steps
    \end{itemize}
    \item \textbf{Combined Vulnerability Testing}:
    \begin{itemize}
        \item Identify DOM XSS vulnerabilities
        \item Combine with clickjacking vectors
        \item Test for privilege escalation scenarios
    \end{itemize}
\end{enumerate}
\end{tcolorbox}


\subsection*{Testing Checklist Summary}

\begin{tcolorbox}[title=Clickjacking Assessment Checklist]
\begin{itemize}
    \item \textbf{Reconnaissance}
    \begin{itemize}
        \item Map sensitive functionalities and endpoints
    \end{itemize}
    \item \textbf{Header Analysis}
    \begin{itemize}
        \item Check for \texttt{X-Frame-Options} header
        \item Check for \texttt{CSP frame-ancestors} directive
        \item Verify header consistency across pages
    \end{itemize}
    \item \textbf{Vulnerability Confirmation}
    \begin{itemize}
        \item Basic iframe loading test
        \item Browser extension verification
        \item Burp Clickbandit assessment
    \end{itemize}
    \item \textbf{Exploitation Testing}
    \begin{itemize}
        \item Frame buster bypass attempts
        \item Prefilled form testing
        \item Multi-step action testing
        \item Combined vulnerability testing
    \end{itemize}
\end{itemize}
\end{tcolorbox}

\subsection*{Tools \& Commands Quick Reference}

\begin{tcolorbox}[title=Essential Testing Tools]
\begin{itemize}
    \item \textbf{Header Analysis}: curl, browser dev tools
    \item \textbf{Visual Indicators}: Firefox "Click-jacking" extension
    \item \textbf{PoC Generation}: Burp Clickbandit
    \item \textbf{Manual Testing}: Custom HTML files
\end{itemize}
\end{tcolorbox}

\begin{lstlisting}[title=Key Commands, basicstyle=\ttfamily\small]
# Header checking
curl -I https://target.com | grep -i frame

# Bulk testing multiple endpoints
cat urls.txt | while read url; do
    echo "Testing: $url"
    curl -I "$url" | grep -i "x-frame-options\|content-security-policy"
done

# Quick manual test
echo '<iframe src="URL" style="opacity:0.5;"></iframe>' > test.html
\end{lstlisting}

\newpage

\section*{Remediation and Prevention Measures}

\subsection*{Quick Fix: Add Security Headers}

Add these headers to your web server configuration:

\begin{tcolorbox}[title=For Apache (.htaccess)]
\begin{lstlisting}
# Block ALL framing (most secure)
Header always set X-Frame-Options "DENY"
Header always set Content-Security-Policy "frame-ancestors 'none'"

# OR allow same-site only
Header always set X-Frame-Options "SAMEORIGIN"
Header always set Content-Security-Policy "frame-ancestors 'self'"
\end{lstlisting}
\end{tcolorbox}

\begin{tcolorbox}[title=For Nginx (server config)]
\begin{lstlisting}
# Block ALL framing
add_header X-Frame-Options "DENY" always;
add_header Content-Security-Policy "frame-ancestors 'none'" always;

# OR allow same-site only  
add_header X-Frame-Options "SAMEORIGIN" always;
add_header Content-Security-Policy "frame-ancestors 'self'" always;
\end{lstlisting}
\end{tcolorbox}

\subsection*{For Developers: Code Solutions}

\begin{tcolorbox}[title=PHP]
\begin{lstlisting}[language=PHP]
<?php
// Add to top of sensitive pages
header('X-Frame-Options: DENY');
header("Content-Security-Policy: frame-ancestors 'none'");
?>
\end{lstlisting}
\end{tcolorbox}

\begin{tcolorbox}[title=Node.js/Express]
\begin{lstlisting}
// For all routes
app.use((req, res, next) => {
    res.setHeader('X-Frame-Options', 'DENY');
    res.setHeader("Content-Security-Policy", "frame-ancestors 'none'");
    next();
});
\end{lstlisting}
\end{tcolorbox}

\begin{tcolorbox}[title=ASP.NET]
\begin{lstlisting}
// In Web.config
<system.webServer>
  <httpProtocol>
    <customHeaders>
      <add name="X-Frame-Options" value="DENY" />
      <add name="Content-Security-Policy" value="frame-ancestors 'none'" />
    </customHeaders>
  </httpProtocol>
</system.webServer>
\end{lstlisting}
\end{tcolorbox}

\subsection*{Simple Testing}

Check if your fix works:

\begin{tcolorbox}[title=Quick Test Command]
\begin{lstlisting}[language=bash]
curl -I https://yoursite.com | grep -i "frame"
\end{lstlisting}
\end{tcolorbox}

You should see:
\begin{verbatim}
X-Frame-Options: DENY
Content-Security-Policy: frame-ancestors 'none'
\end{verbatim}

\subsection*{What to Protect}

Add headers to these pages:
\begin{itemize}
    \item Login pages
    \item Admin panels  
    \item Payment pages
    \item Account settings
    \item Any page with sensitive actions
\end{itemize}

\subsection*{Remember}

\begin{tcolorbox}[title=Key Points]
\begin{itemize}
    \item Use \texttt{DENY} or \texttt{SAMEORIGIN} for X-Frame-Options
    \item Use \texttt{'none'} or \texttt{'self'} for CSP frame-ancestors
    \item Test with browser tools or curl command
    \item Apply to ALL sensitive pages
    \item Both headers provide extra protection
\end{itemize}
\end{tcolorbox}



\end{document}