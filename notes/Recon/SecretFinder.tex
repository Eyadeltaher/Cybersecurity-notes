\documentclass{article}
\usepackage{listings}
\usepackage{xcolor}
\usepackage{hyperref}


\title{Secret Finding in Web Apps}
\author{Eyad El-Taher}
\date{\today}


\definecolor{codegreen}{rgb}{0,0.6,0}
\definecolor{codegray}{rgb}{0.5,0.5,0.5}
\definecolor{codepurple}{rgb}{0.58,0,0.82}
\definecolor{backcolour}{rgb}{0.95,0.95,0.92}

\lstdefinestyle{mystyle}{
    backgroundcolor=\color{backcolour},   
    commentstyle=\color{codegreen},
    keywordstyle=\color{magenta},
    numberstyle=\tiny\color{codegray},
    stringstyle=\color{codepurple},
    basicstyle=\ttfamily\footnotesize,
    breakatwhitespace=false,         
    breaklines=true,                 
    captionpos=b,                    
    keepspaces=true,                 
    numbers=left,                    
    numbersep=5pt,                  
    showspaces=false,                
    showstringspaces=false,
    showtabs=false,                  
    tabsize=2
}

\lstset{style=mystyle}
\begin{document}

\maketitle

\section{Introduction}
Finding misconfigured JavaScript (JS) files in web applications is an important security testing methodology. This vulnerability discovery consists of two main parts: first, finding all JS files of a target website, and second, extracting sensitive information from those JS files.

\section{Step 1: Finding JS Files}
To find all JavaScript files of a target website, we use Katana, a web crawling framework. The following command enumerates JS files from a list of domains:

\begin{lstlisting}[language=bash, caption={Command to find JS files using Katana}]
katana -list {domains.txt} -d 5 -jc | grep ".js$" | uniq | sort | tee js_files.txt
\end{lstlisting}

\subsection{Command Explanation}
\begin{itemize}
    \item \texttt{katana -list {domains.txt}}: Use Katana with a list of target domains
    \item \texttt{-d 5}: Set crawling depth to 5
    \item \texttt{-jc}: Enable JavaScript file crawling
    \item \texttt{grep ".js\$"}: Filter for files ending with .js extension
    \item \texttt{uniq}: Remove duplicate entries
    \item \texttt{sort}: Sort the output alphabetically
    \item \texttt{| tee js\_files.txt}: Shows output on screen AND saves to file
    \item The saved file \texttt{js\_files.txt} will contain all discovered JS file URLs
\end{itemize}

\textbf{Note:} Replace \texttt{\{domains.txt\}} with your actual file containing subdomains of the target website.

\section{Step 2: Extracting Sensitive Information}
After obtaining the list of JS files, we use \textbf{SecretFinder} to extract sensitive information from these files.


\subsection{Running SecretFinder}
Execute the following command to scan all JS files:
\begin{lstlisting}[language=bash, caption={Command to extract secrets from JS files}]
cat js_files.txt | while read url; do 
    SecretFinder -i $url -o cli; 
done > secrets_found.txt
\end{lstlisting}

\subsection{Command Explanation}
\begin{itemize}
    \item \texttt{cat {jsfilesgottenfromkatana.txt}}: Read the file containing JS URLs
    \item \texttt{while read url; do ... done}: Process each URL line by line
    \item \texttt{python3 SecretFinder/SecretFinder.py}: Run the SecretFinder tool
    \item \texttt{-i \$url}: Input URL to scan
    \item \texttt{-o cli}: Output results to command line interface
\end{itemize}

\textbf{Note:} Replace \texttt{\{jsfilesgottenfromkatana.txt\}} with the actual filename containing your discovered JS files.




\end{document}