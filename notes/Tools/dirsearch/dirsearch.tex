\documentclass{article}
\usepackage[utf8]{inputenc}
\usepackage[T1]{fontenc}
\usepackage{geometry}
\usepackage{listings}
\usepackage{xcolor}
\usepackage{enumitem}
\usepackage{hyperref}

\geometry{a4paper, left=15mm, right=15mm, top=20mm, bottom=20mm}

\title{Dirsearch Tool Notes}
\author{Web Path Brute-Forcer}
\date{}

\begin{document}

\maketitle

\section*{Important Explanations}

\subsection*{Tool Overview}
\begin{itemize}[leftmargin=*]
    \item \textbf{Dirsearch}: Web path brute-forcer to discover directories and files on web servers
    \item \textbf{Default Settings}: 25 threads, HTTP GET method, no recursion, no random user-agents
    \item \textbf{Configuration}: Default config loaded from \texttt{config.ini}, override with \texttt{--config} or \texttt{DIRSEARCH\_CONFIG} environment variable
\end{itemize}

\subsection*{Wordlist Handling}
\begin{itemize}[leftmargin=*]
    \item \texttt{\%EXT\%} keyword in wordlist entries is replaced with extensions passed via \texttt{-e} flag
    \item \texttt{-f / --force-extensions}: Append extensions (and "/") to every entry if wordlist lacks \texttt{\%EXT\%}
    \item \texttt{-O / --overwrite-extensions}: Force replacement of existing extensions in wordlist entries
    \item \texttt{-X / --exclude-extensions}: Exclude certain extensions from wordlist
    \item \textbf{Wordlist case support}: lowercase, uppercase, capitalization
\end{itemize}

\subsection*{Recursion Features}
\begin{itemize}[leftmargin=*]
    \item \texttt{-r / --recursive}: Continue brute-forcing inside discovered directories
    \item \texttt{--max-recursion-depth}: Limit recursion depth
    \item \texttt{--recursion-status}: Specify status codes for recursion
    \item \textbf{Special recursion modes}:
    \begin{itemize}
        \item \textbf{Force-recursive}: Brute-force recursively all found paths
        \item \textbf{Deep-recursive}: Brute-force all sub-depths (e.g., for \texttt{a/b/c}, also scan \texttt{a/}, \texttt{a/b/})
    \end{itemize}
    \item \texttt{--exclude-subdirs}: Exclude subdirectories from recursion
    \item \texttt{--subdirs}: Scan specific subdirectories
\end{itemize}

\subsection*{Fuzzing Techniques}
\begin{itemize}[leftmargin=*]
    \item \texttt{--prefixes}: Add prefixes to each wordlist entry (useful for \texttt{.backup}, etc.)
    \item \texttt{--suffixes}: Add suffixes to each wordlist entry (useful for \texttt{\~}, etc.)
\end{itemize}

\subsection*{Filtering \& Blacklisting}
\begin{itemize}[leftmargin=*]
    \item \textbf{Blacklist files}: Located in \texttt{db/} for status codes; matching paths are filtered out
    \item \textbf{Filter options}:
    \begin{itemize}
        \item \texttt{--include-status}, \texttt{--exclude-status}: Status code filters
        \item \texttt{--exclude-sizes}: Response size filters
        \item \texttt{--exclude-texts}: Response text filters
        \item \texttt{--exclude-regexps}: Regex filters
        \item \texttt{--exclude-redirects}: Redirect pattern filters
        \item \texttt{--exclude-response}: Specific response filters
    \end{itemize}
\end{itemize}

\subsection*{Advanced Features}
\begin{itemize}[leftmargin=*]
    \item \texttt{--raw}: Import raw HTTP request from file
    \item \texttt{--scheme}: Set scheme if dirsearch cannot guess correct one
    \item \texttt{--proxy / --proxy-list}: HTTP \& SOCKS proxy support
    \item \texttt{--format}: Multiple report formats (simple, plain, json, xml, md, csv, html, sqlite)
    \item \texttt{-o}: Save output to file
    \item \textbf{Pause/Resume}: Stop scans via \texttt{Ctrl+C} and resume or skip targets/subdirectories
\end{itemize}

\section*{Commands}

\subsection*{Basic Scanning}
\begin{lstlisting}[basicstyle=\ttfamily\small, frame=single]
# Basic scan with default extensions
dirsearch -u https://target

# Scan with specific extensions
dirsearch -e php,html,js -u https://target

# Scan with custom wordlist
dirsearch -e php,html,js -u https://target -w /path/to/wordlist
\end{lstlisting}

\subsection*{Recursive Scanning}
\begin{lstlisting}[basicstyle=\ttfamily\small, frame=single]
# Basic recursive brute-force
dirsearch -e php,html,js -u https://target -r

# Limited recursive scan (depth 3, status 200-399)
dirsearch -e php,html,js -u https://target -r --max-recursion-depth 3 --recursion-status 200-399
\end{lstlisting}

\subsection*{Advanced Fuzzing}
\begin{lstlisting}[basicstyle=\ttfamily\small, frame=single]
# Add prefixes to wordlist entries
dirsearch -e php -u https://target --prefixes .,admin,_

# Add suffixes for backup files
dirsearch -e php -u https://target --suffixes ~
\end{lstlisting}

\section*{Important Notes}

\subsection*{Performance \& Optimization}
\begin{itemize}[leftmargin=*]
    \item \textbf{Thread warning}: Too many threads risk DoS; default is 25 but adjustable
    \item \textbf{Slow servers}: Use HEAD method instead of GET to reduce time
    \item \texttt{--skip-on-status 429}: Skip when server responds with "Too Many Requests"
    \item \textbf{Bypass techniques}: Randomize user-agent or use proxy list to bypass request limits or detection
\end{itemize}

\subsection*{Best Practices}
\begin{itemize}[leftmargin=*]
    \item Use \texttt{--force-extensions} when wordlist doesn't include \texttt{\%EXT\%}
    \item Use \texttt{--overwrite-extensions} carefully - some extensions may not get overwritten
    \item Combine include/exclude filters for noisy or irrelevant results
    \item For hidden config files or backups: use \texttt{--prefixes .} and \texttt{--suffixes \~}
\end{itemize}

\end{document}