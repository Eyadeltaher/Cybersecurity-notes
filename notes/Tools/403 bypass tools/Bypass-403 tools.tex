\documentclass{article}
\usepackage[utf8]{inputenc}
\usepackage{hyperref}
\usepackage{listings}
\usepackage{xcolor}

\lstset{
    backgroundcolor=\color{gray!10},
    basicstyle=\ttfamily\small,
    breaklines=true,
    frame=single,
    keywordstyle=\color{blue}
}

\title{Analysis of HTTP 403 Bypass Tools}
\author{Eyad El-Taher}
\date{\today}

\begin{document}

\maketitle

\section{Tool 1: Bypass-403}
\textbf{GitHub:} \url{https://github.com/iamj0ker/bypass-403}

\subsection{Description}
A lightweight shell script designed for security researchers to bypass 403 Forbidden restrictions. It allows users to compare server responses under various conditions to identify misconfigurations.

\subsection{Installation}
To set up the environment and dependencies:
\begin{lstlisting}[language=bash]
git clone https://github.com/iamj0ker/bypass-403
cd bypass-403
chmod +x bypass-403.sh
sudo apt install figlet jq
\end{lstlisting}

\subsection{Usage}
\begin{lstlisting}[language=bash]
./bypass-403.sh https://example.com admin
./bypass-403.sh [website] [path]
\end{lstlisting}

\subsection{Features}
\begin{itemize}
    \item Utilizes 24 known bypass techniques.
    \item Leverages \texttt{curl} for request automation.
    \item \textbf{Contributors:} remonsec, manpreetMayankPandey01, saadibabar.
\end{itemize}

\newpage

\section{Tool 2: 4-ZERO-3}
\textbf{GitHub:} \url{https://github.com/Dheerajmadhukar/4-ZERO-3}

\subsection{Introduction}
4-ZERO-3 is a comprehensive script designed to bypass 403/401 unauthorized errors. It incorporates a wide array of techniques and provides the exact \texttt{cURL} payload when a successful bypass is detected.

\begin{quote}
    \textbf{Note on False Positives:} If multiple [200 OK] responses are received, verify the \texttt{Content-Length}. Identical lengths often indicate a false positive caused by redirects (301/302).
\end{quote}

\subsection{Operational Modes}
The tool supports specific flags for targeted testing:

\begin{itemize}
    \item \textbf{Protocol:} \texttt{--protocol} (Protocol-based payloads)
    \item \textbf{Port:} \texttt{--port} (Port-based payloads)
    \item \textbf{HTTP Methods:} \texttt{--HTTPmethod} (Verb tampering)
    \item \textbf{Encoding:} \texttt{--encode} (URL encoded bypasses)
    \item \textbf{SQLi:} \texttt{--SQLi} (Mod\_Security \& libinjection bypasses)
    \item \textbf{Headers:} \texttt{--header} (Header-based manipulation)
    \item \textbf{Complete scan:} \texttt{--exploit}
\end{itemize}

\subsection{Usage Examples}
Execute a complete scan against an endpoint:
\begin{lstlisting}[language=bash]
bash 403-bypass.sh -u https://target.com/secret --exploit
\end{lstlisting}

\subsection{Prerequisites}
\begin{itemize}
    \item \texttt{apt install curl} (Debian-based systems)
\end{itemize}

\end{document}